\chapter{Use case}
\label{cha:789}
To test the operation of the previously exposed implementation, we selected an example dataset generated through a script implemented by Microsoft Azure. This choice was made due to the scarcity of online datasets that were complete enough to obtain an acceptable result. This scarcity is likely due to the sensitivity of the information contained in them.
\section{Sensor Data Generator}
The data generator is an open-source physics-inspired simulation framework that can be customised for the generation of data for model training. Generated data are a combination of telemetry and maintenance records.\\
Generated telemetry data contains:
\begin{itemize}
	\item Timestamp
	\item MachineID
	\item Ambient pressure
	\item Ambient temperature
	\item Rotational speed
	\item Desired rotational speed
	\item Temperature
	\item Pressure
\end{itemize}
Maintenance generated record contains:
\begin{itemize}
	\item Timestamp
	\item MachineID
	\item Level (INFO, WARNING, ERROR, CRITICAL)
	\item Code (identifies event/failure type)
\end{itemize}
This data does not represent a particular machinery but it is supposed that it could represent rotational motors or hydraulic pumps.\\
\section{Data Preparation}
\subsection{Data Ingestion}
In order to align the generated dataset with the data collection architecture depicted above, preliminary operations are necessary. First, a suitable structure must be created in the system. This involved the creation of a deployment, a zone, and two machines with six sensors each. Next, the generated data must be divided by sensor. This is to simulate sending the data by individual sensor. This is done by going to create records containing the timestamp, the value read from the sensor, and the sensor ID. Then, the data must be sent on the MQTT topic where the system listens. This allows the data to be uploaded to Timescale.
\subsection{Feature Engineering}
In the context of machine learning, a feature is a single measurable attribute of a physical phenomenon; the process of extracting features from raw data is called feature engineering.\\
In this case, the data is aggregated into machine cycles, which are periods of time when the machine is in a particular state; for example, the operation of an electric motor can be divided into cycles based on the states of on, rotating, and off. \\
This aggregation is necessary because the raw data cannot be used directly to train a predictive maintenance model. %Capire il perch� � necessario aggregare i dati
\\
Aggregation into cycles usually works because it is rather unlikely that there will be a sudden degradation during an operating cycle.\\
Cycles are dynamic in that they do not have a fixed duration, but adapt to the duration of the machine cycle, separating cycles when there is an interruption of at least 30 seconds in the data.\\
In addition, it is necessary to merge maintenance type data with telemetry type data aggregated by time and machine. This allows us to go in and calculate the Remaining Useful Life (RUL) based on the position of the various telemetry data in the sequence; this way we get the number of cycles for which the machine will continue to operate smoothly. For example, the cycle just before a problem will have a RUL equal to 1, the cycle before that will have a RUL equal to 2, and so on.\\ %Aggiungere data augmentation con rolling window nei future works
These operations to group the data for training were done through a complex query on Timescale, which allows us to get a data set composed in this way.
\section{Model Training}
In order to train the model, a number of preliminary operations were performed. Initially, a query was issued to the database, as previously described, and the data was then stored in Pandas data frames. These are a highly efficient two-dimensional data structure commonly used in data science. They were selected because they provide a pre-defined set of data manipulation operations and are highly efficient data structures.\\
Subsequently, sensor datasets belonging to the same machine with the same timestamp were merged into a single row, resulting in a structure that contains
\begin{itemize}
	\item asset\_id
	\item cycle
	\item speed\_desired\_max
	\item speed\_avg
	\item temperature\_avg
	\item temperature\_max
	\item pressure\_avg
	\item pressure\_max
	\item rul
\end{itemize}
Subsequently, the dataset is partitioned into two distinct data structures: "sequences" and "labels." The sequences are two-dimensional NumPy arrays that encompass all data except the rule, while the labels are NumPy arrays that exclusively contain the rule. This process results in the creation of two subdatasets, with the sequences serving as the input data for the model and the labels representing the output data that the model is tasked with predicting. The selection of a NumPy array as the data structure was driven by the necessity of aligning with the requirements of the utilised model.
\subsection{Features Scaling}
The utilisation of a deep learning model, such as LSTM, necessitates the normalisation of features. This optimises the training of the model, resulting in enhanced speed and accuracy.\\
Specifically, the normalisation of features facilitates a faster and more efficient convergence of the gradient. This, in turn, influences the training of the LSTM model, which exploits the descending vanishing problem in order to 'learn'. Furthermore, neural networks are susceptible to numerical instability issues that can be mitigated by scaling the values to eliminate those that are excessively large or small. Finally, feature normalisation enhances the model's generalisability by improving its ability to learn from diverse data.\\
In our case, we employed an object provided by the "Scikit-Learn" library called "MinMaxScaler." This object scales the data in a range from zero to one by applying a linear scaling transformation. It should be noted that this normalisation technique does not reduce the effects of outliers, as the smallest value in the dataset corresponds to 0 and the largest value in the dataset corresponds to 1.\\
The scaled values are obtained with the following formula:
\[X_{scaled} = \frac{X - X_{min}}{X_{max} - X_{min}}\]
Where X is the value that has to be scaled, $X_{min}$ is the smallest value in the dataset and $X_{max}$ is the biggest value in the dataset.
\subsection{Dataset Splitting}
\subsection{Model composition}






















